%!TEX root = ../main.tex
\setcounter{chapter}{7}
\setcounter{section}{1}
\section{Dynamic Analysis of Discrete Systems}
\vspace{-8pt} \hrule \hrule \hrule \hrule \hrule  \vspace{12pt}


\begin{table}[!h]
\centering

\begin{tabular}{|cccc|}
\hline

\multicolumn{2}{|c|}{Continuous-Time}                                      & \multicolumn{2}{c|}{Discrete-Time}   \\ \hline
\multicolumn{1}{|c|}{periodic}              & \multicolumn{1}{c|}{Discrete Fourier Series}         & \multicolumn{1}{c|}{periodic}            & Fourier Series    \\ \hline
\multicolumn{1}{|c|}{absolutely integrable} & \multicolumn{1}{c|}{Discrete-Time Fourier Transform} & \multicolumn{1}{c|}{absolutely summable} & Fourier Transform \\ \hline
\multicolumn{1}{|c|}{causal}                & \multicolumn{1}{c|}{z-Transform}                     & \multicolumn{1}{c|}{causal}              & Laplace Transform \\ \hline
\end{tabular}
\end{table}
\begin{itemize}
	\item $z$-transform for discrete time systems $\leftrightarrow$ Laplace transform for continuous time systems. 
\item (8.2.1) $z$-Transform 
	\begin{enumerate}
		\item Laplace transform and its important property 
		\begin{align*}
			\mathcal{L} (f(t))&= F(s) = \int_{0^-}^{\infty} f(t) e^{-st} dt 
			&&&
			\mathcal{L}(\dot{f}(t)) &= sF(s) -f(0^{-})\\
			&&\Downarrow&
			\\
						\mathcal{L} (f(t))&= F(s) = \int_0^{\infty} f(t) e^{-st} dt 
			&&&
			\mathcal{L}(\dot{f}(t)) &= sF(s) \text{ } \text{ where $f(0^+) = 0$}
		\end{align*}
		$0^{-}$부터인 이유는 $f(t)$가 $\delta(t)$나 $\frac{d\delta(t)}{dt}$일때 Laplace Transform에 반영하기 위함, 이해를 돕기위해 정확한 정의는 아니지만 아래와 같은 정의 사용
	\end{enumerate}	
\end{itemize}

