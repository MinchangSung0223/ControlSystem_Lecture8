%!TEX root = ../main.tex
\setcounter{chapter}{7}
\setcounter{section}{1}
\section{Dynamic Analysis of Discrete Systems}
\vspace{-8pt} \hrule \hrule \hrule \hrule \hrule  \vspace{12pt}

% Please add the following required packages to your document preamble:
% \usepackage{booktabs}

\begin{table}[!hb]
\centering
\begin{tabular}{@{}|c|c|c|@{}}
\toprule
Condition                           & Discrete-Time                   & Continuous-Time   \\ \midrule
Periodic Signal                     & Discrete Fourier Series         & Fourier Series    \\ \midrule
Absolute Summable/Integrable Signal & Discrete-Time Fourier Transform & Fourier Transform \\ \midrule
Causal Signal                       & Unilateral $z$-Transfrom          & Unilateral Laplace Transform \\ \bottomrule
\end{tabular}
\end{table}
\begin{itemize}
	\item $z$-transform for discrete time systems $\leftrightarrow$ Laplace transform for continuous time systems. 
\item (8.2.1) $z$-Transform 
	\begin{enumerate}
		\item Laplace transform and its important property 
		\begin{align*}
			\mathcal{L} (f(t))&= F(s) = \int_{0^-}^{\infty} f(t) e^{-st} dt 
			&&&
			\mathcal{L}(\dot{f}(t)) &= sF(s) -f(0^{-})\\
			&&\Downarrow&
			\\
						\mathcal{L} (f(t))&= F(s) = \int_0^{\infty} f(t) e^{-st} dt 
			&&&
			\mathcal{L}(\dot{f}(t)) &= sF(s) \text{ } \text{ where $f(0^+) = 0$}
		\end{align*}
		$0^{-}$부터인 이유는 $f(t)$가 $\delta(t)$나 $\frac{d\delta(t)}{dt}$일때 Laplace Transform에 반영하기 위함, 이해를 돕기위해 정확한 정의는 아니지만 아래와 같은 정의 사용
	\end{enumerate}	
\end{itemize}

