%!TEX root = ../main.tex
\setcounter{chapter}{7}
\setcounter{section}{0}

%
\section{Digitization}
\vspace{-8pt} \hrule \hrule \hrule \hrule \hrule  \vspace{12pt}


\begin{enumerate}\addtocounter{enumi}{6}
 \item The D/A converter changes the digital binary number to an analog voltage, and a zeroth-order hold maintains the same voltage throughout the sample period $T$. (DAC and ZOH)

 \begin{itemize}
  \item Because each value of $u(kT)$ in Fig. 8.1(b) is held constant until the next value is available from the computer, the continuous value of $u(t)$ consists of steps (see Fig. 8.2) that, on average, are delayed from a fit to $u(kT)$ by $T/2$ as shown in the figure. 
  \item Sample rates should be at least 20 times the bandwidth in order to assure that the digital controller will match the performance of the continuous controller.
  \item If we simply incorporate this $T/2$ delay into a continuous analysis of the system, an excellent prediction results in, especially, for sample rates much slower than 20 times bandwidth.
 \end{itemize}
 \item A system having both discrete and continuous signals is called a `sampled data system'. 

\end{enumerate}

