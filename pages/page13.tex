%!TEX root = ../main.tex
\setcounter{chapter}{7}
\setcounter{section}{0}
\section{Digitization}
\vspace{-8pt} \hrule \hrule \hrule \hrule \hrule  \vspace{12pt}

	\begin{enumerate}	

	\item $z$-transform is defined by 
		\begin{align*}
			\mathcal{Z}(f(k)) &= F(z) = \sum_{k=0}^{\infty} f(k) z^{-k}  
			&&&
			\mathcal{Z} (f(k-1)) &=   \sum_{k=0}^{\infty} f(k-1) z^{-k}  
			\\
			&= f(0) + f(1) z^{-1} + f(2) z^{-2} + \cdots
			&&&
			&= f(-1) + f(0) z^{-1} + f(1) z^{-2} + f(2) z^{-3} + \cdots \\
			& &&& &= z^{-1} \left[  f(0) + f(1) z^{-1} + f(2) z^{-2} + \cdots \right] \\ 
			& &&& &= z^{-1} F(z) 
		\end{align*}
		where $f(k)$ is the sampled version of $f(t)$ and $z^{-1}$ represents one sample delay, and $f(-1) = 0$. 
		\item Important property between LT and $z$-transform
		\begin{align*}
			z = e^{sT} ~~~~ \leftrightarrow~~~~ s = \frac{1}{T} \ln z  
		\end{align*}
		\item For example, the general second-order difference equation 
		\begin{align*}
			y(k) = -a_1 y(k-1) - a_2 y(k-2) + b_0 u(k) + b_1 u(k-1) + b_2 u(k-2) 
		\end{align*}
		can be converted from this form to the $z$-transform of the variables $y(k)$ and $u(k)$ by invoking above relations,
		\begin{align*}
			Y(z) = (-a_1 z^{-1} - a_2 z^{-2}) Y(z) + (b_0 + b_1 z^{-1} + b_2 z^{-2}) U(z) 
		\end{align*}
		now we have a discrete transfer function:
		\begin{align*}
			\frac{Y(z)}{U(z)} = \frac{b_0 + b_1 z^{-1} + b_2 z^{-2}}{1 + a_1 z^{-1} + a_2 z^{-2}} 
		\end{align*}
	\end{enumerate}	
