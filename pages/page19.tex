%!TEX root = ../main.tex

\setcounter{chapter}{7}
\setcounter{section}{1}
\section{Dynamic Analysis of Discrete Systems}
\vspace{-8pt} \hrule \hrule \hrule \hrule \hrule  \vspace{12pt}
\begin{itemize}
\item (8.2.3) Relationship between $s$ and $z$ 
\begin{enumerate}
		\item Consider the continuous signal of 
		\begin{align*}
			f(t) &= e^{-at} ~~~~~ t >0  ~~~~~ \rightarrow ~~~~~  F(s) = \frac{1}{s+a}
		\end{align*}
		and it corresponds to a pole $s=-a$. 
		\item Consider the discrete signal of 
		\begin{align*}
			f(kT) &= e^{-akT}  ~~~~~ \rightarrow ~~~~~  F(z) =  \frac{1}{1-e^{-aT}z^{-1}} = \frac{z}{z-e^{-aT}}
		\end{align*}
		and it corresponds to a pole $z=e^{-aT}$. 
		\item The equivalent characteristics in the $z$-plane are related to those in the $s$-plane by the expression
		\begin{align*}
			z &= e^{sT}\\
			&= e^{-aT +jbT} = e^{-aT} (\cos bT + j \sin b) \\
			&= e^{-\sigma T} (\cos \omega_d T + j \sin \omega_d T ) \\
			&= e^{-\zeta \omega_n T} (\cos \omega_n \sqrt{1-\zeta^2} T + j \sin \omega_n \sqrt{1-\zeta^2} T )
		\end{align*}
		where $T$ is the sample period, and $s = -\sigma  + j \omega_d = -\zeta \omega_n + j \omega_n \sqrt{1-\zeta^2} $
	\end{enumerate}
\end{itemize}

