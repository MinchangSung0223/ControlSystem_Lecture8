%!TEX root = ../main.tex
\setcounter{chapter}{7}
\setcounter{section}{1}
\section{Dynamic Analysis of Discrete Systems}
\vspace{-8pt} \hrule \hrule \hrule \hrule \hrule  \vspace{12pt}

	\begin{enumerate}	
		\setcounter{enumi}{1}



	\item $z$-transform is defined by 
		\begin{align*}
			\mathcal{Z}(f(k)) &= F(z) = \sum_{k=0}^{\infty} f(k) z^{-k}  
			&&&
			\mathcal{Z} (f(k-1)) &=   \sum_{k=0}^{\infty} f(k-1) z^{-k}  
			\\
			&= f(0) + f(1) z^{-1} + f(2) z^{-2} + \cdots
			&&&
			&= f(-1) + f(0) z^{-1} + f(1) z^{-2} + f(2) z^{-3} + \cdots \\
			& &&& &= z^{-1} \left[  f(0) + f(1) z^{-1} + f(2) z^{-2} + \cdots \right] \\ 
			& &&& &= z^{-1} F(z) 
		\end{align*}
		where $f(k)$ is the sampled version of $f(t)$ and $z^{-1}$ represents one sample delay, and $f(-1) = 0$. \\

		Example) $x(0) = 0 ,x(1) = 1, x(2) = 2 ,x(3) = 3, x(4) = 4 $		\\
		\begin{flalign}
		 X(z) &= \sum_{k=0}^{\infty} x(k)z^{-k}\\
		      &=x(0) + x(1)z^{-1}+ x(2)z^{-2}+ x(3)z^{-3}+ x(4)z^{-4} \\
		      &= z^{-1}+2z^{-2}+3z^{-3}+4z^{-4}
        \end{flalign}
		\item Important property between LT and $z$-transform
		\begin{align*}
			z = e^{sT} ~~~~ \leftrightarrow~~~~ s = \frac{1}{T} \ln z  
		\end{align*}

	\end{enumerate}	
