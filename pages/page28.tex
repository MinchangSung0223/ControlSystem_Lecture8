%!TEX root = ../main.tex
\setcounter{chapter}{7}
\setcounter{section}{2}
\section{Design using Discrete Equivalents}
\vspace{-8pt} \hrule \hrule \hrule \hrule \hrule  \vspace{12pt}

\begin{itemize}
	\item (8.3.2) Zeroth-Order Hold (ZOH) Method 
	\begin{enumerate}
		\item Tustin's method essentially assumed that the input to the controller varied linearly early between the past sample  and the current sample. 
		\item Another assumption is that the input to the controller remains constant throughout the sample period.  $\rightarrow$ ZOH 
		\item One input sample produces a square pulse of height $e(k)$ that lasts for one sample period $T$. 
		\item For a constant positive step input, $e(k)$, at time $k$, $E(s) = e(k)/s$, so the result would be
		\begin{align*}
			D_d(z) = \mathcal{Z} \left( \frac{D_c(s)}{s} \right)
		\end{align*}
		Furthermore, a constant negative step, one cycle delayed, would be 
		\begin{align*}
			D_d(z) = z^{-1} \mathcal{Z} \left( \frac{D_c(s)}{s} \right)
		\end{align*}
		Therefore, the discrete TF for the square pulse is 
		\begin{align*}
			D_d(z) = (1-z^{-1}) \mathcal{Z} \left( \frac{D_c(s)}{s} \right)
		\end{align*}	
	\end{enumerate}
\end{itemize}