%!TEX root = ../main.tex
\setcounter{chapter}{7}
\setcounter{section}{1}
\section{Dynamic Analysis of Discrete Systems}
\vspace{-8pt} \hrule \hrule \hrule \hrule \hrule  \vspace{12pt}
\begin{itemize}
\item (8.2.2) $z$-Transform Inversion
	\begin{enumerate}


		\item A $z$-transform inversion technique that has no continuous counterpart is called `long division'. For example, consider a first-order discrete system 
		\begin{align*}
			y(k) = \alpha y(k-1) + u(k) ~~~~\rightarrow~~~~~ Y(z) = \alpha z^{-1}Y(z)+U(z)  ~~~~\rightarrow~~~~~  \frac{Y(z)}{U(z)} = \frac{1}{1-\alpha z^{-1}}
		\end{align*} 
		For a unit-pulse input, its $z$-transform is 
		\begin{align*}
			U(z) = \mathcal{Z}(\delta(kT)) = 1 
		\end{align*}
		so the long division becomes 
		\begin{align*}
			Y(z) &= \frac{1}{1-\alpha z^{-1}} \\
			&= 1 + \alpha z^{-1} + \alpha^2 z^{-2} + \alpha^3 z^{-3}  \cdots  		
		\end{align*}

		We see that the sampled time history of $y$ is
		\begin{align*}
			y(0) &= 1 &&& y(1) &= \alpha &&& y(2) &= \alpha^2 &&& y(3) &= \alpha^3 ~~~~\cdots 
		\end{align*}
	\end{enumerate}		
\end{itemize}