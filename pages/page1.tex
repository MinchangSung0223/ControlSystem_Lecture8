%!TEX root = ../main.tex
\setcounter{chapter}{7}
\setcounter{section}{0}
\section{Digitization}
\vspace{-8pt} \hrule \hrule \hrule \hrule \hrule  \vspace{12pt}
\begin{enumerate}
 \item Most control systems use digital computers (usually microprocessors) to implement the controller. 
 \item Sampler and A/D Converter, D/A Converter and ZOH (Zeroth-Order Holding), and Clock
 \item The computation of error signal $e(t)$ and the dynamic compensation $D_c(s)$ can all be accomplished in a digital computer. 
 \item Difference equation for discrete-time system $\leftrightarrow$ Differential equation for continuous-time system 
 \item Two basic techniques for finding the difference equations for the digital controller, from $D_c(s)$ to $D_d(z)$ 
 \begin{itemize}
  \item Discrete equivalent - section 8.3 
  \item Discrete design  - section 8.7 
 \end{itemize} 
 \item The analog output of the sensor is sampled and converted to a digital number in the analog-to-digital (A/D) converter. (Sampler and ADC)
 \begin{itemize}
  \item Conversion from the continuous analog signal $y(t)$ to the discrete digital samples $y(kT)$ occurs repeatedly at instants of time $T$ apart where $T$ is the sample period [$s$] and $1/T$ is the sample rate [$Hz$].
  \begin{align*}
   y(t) ~~~~ \rightarrow~~~~~y (k) = y(kT) ~~~~\mbox{with} ~~ t = kT  
  \end{align*}
  where $k$ is an integer and $T$ is a fixed value (sample period, or sampling time). 
  \item The sampled signal is $y(kT)$, where $k$ can take on any integer value. 
  \item It is often written simply as $y(k)$. We call this type of variable a discrete signal. 
 \end{itemize}  

\end{enumerate}

