%!TEX root = ../main.tex
\setcounter{chapter}{7}
\setcounter{section}{0}
\section{Digitization}
\vspace{-8pt} \hrule \hrule \hrule \hrule \hrule  \vspace{12pt}
\begin{enumerate}
	\setcounter{enumi}{1}
	\item Generalized function

	\begin{itemize}
			\centering
		\item $\delta(t) = 0$ for $ t \neq 0$ 
		\item $\int_{-\infty}^{\infty} \delta(t) dt = 1$ 
	\end{itemize}
\end{enumerate}

$\bigstar$ Unit step function is mathematically defined as:
\begin{align*}
	1(t) = \begin{cases}1 & t > 0\\ undefined & t = 0 \\0 & t< 0\end{cases}
\end{align*}
In contol theory
\begin{align*}
	1(t) = \begin{cases}1 & t \geq 0 \\0 & t< 0\end{cases}
\end{align*}



$\bigstar$ Useful Properties
 \begin{enumerate}
 	\item $\frac{d1(t)}{dt} = \delta(t) $ (수학적으로는 틀림, 개념적으로 사용)
 	\item $x(t)\delta(t-kT) = x(kT)\delta(t-kT)$
 	\item $\int_{-\infty}^{\infty} x(t) \delta(t-kT) dt= x(kT)$\\
		$ \because \int_{-\infty}^{\infty} x(t) \delta(t-kT)dt =\int_{-\infty}^{\infty} x(kT) \delta(t-kT) dt=x(kT)\int_{-\infty}^{\infty} \delta(t-kT) dt=x(kT)$
 	
 
 \end{enumerate}


