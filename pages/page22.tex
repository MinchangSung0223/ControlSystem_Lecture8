%!TEX root = ../main.tex
\setcounter{chapter}{7}
\setcounter{section}{2}
\section{Design using Discrete Equivalents}
\vspace{-8pt} \hrule \hrule \hrule \hrule \hrule  \vspace{12pt}
\begin{itemize}
	\item 1장부터 7장까지 continuous compensation $D_c(s)$를 설계하는 법에 대해서 배웠으며, 이번 절에서는 $D_c(s)$를 discrete compensation $D_d(z)$로 변환하는 방법들과 각 방법들의 성능에 대해서 배운다.
	\item 다만, 이 방법들은 모두 근사법이며 완벽하게 $D_c(s)$와 동일한 성능을 갖는 $D_d(z)$를 설계할 수는 없다.
	\item 일반적인 $D_c(s)$는 Differential equation으로 표현되며 이에 대응하는 $D_d(z)$는 Difference equation으로 표현하게 된다.
	\item (8.3.1 - 8.3.4), (Tustin,ZOH,MPZ,MMPZ)

	
	
	
\end{itemize}
