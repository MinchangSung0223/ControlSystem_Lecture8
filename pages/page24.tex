%!TEX root = ../main.tex
\setcounter{chapter}{7}
\setcounter{section}{2}
\section{Design using Discrete Equivalents}
\vspace{-8pt} \hrule \hrule \hrule \hrule \hrule  \vspace{12pt}

\begin{enumerate}
	\setcounter{enumi}{1}
		\item Taking $z$-transform, 
		\begin{align*}
			\frac{U(z)}{E(z)} = \frac{T}{2} \frac{1+z^{-1}}{1-z^{-1}} = \frac{1}{\frac{2}{T} \frac{1-z^{-1}}{1+z^{-1}} }
		\end{align*}
		\item In fact, the Tustin's method approximates $z = e^{sT}$ as follows:
		\begin{align*}
			s \approx  \frac{2}{T} \frac{1-z^{-1}}{1+z^{-1}}
		\end{align*}
		where it can be derived from the Taylor's series expansions as follows:
		\begin{align*}
			z &= e^{sT} = \frac{e^{\frac{sT}{2}}}{e^{-\frac{sT}{2}}} 
			= \frac{1 + \frac{sT}{2} + \frac{s^2T^2}{2^2} + \cdots}{1 - \frac{sT}{2} + \frac{s^2T^2}{2^2} - \cdots} \approx  \frac{1 + \frac{sT}{2}}{1 - \frac{sT}{2}}  = \frac{2+sT}{2-sT} 
			~~~~~\rightarrow~~~~~ s \approx  \frac{2}{T} \frac{z-1}{z+1} = \frac{2}{T} \frac{1-z^{-1}}{1+z^{-1}}
		\end{align*}

\end{enumerate}